% Options for packages loaded elsewhere
\PassOptionsToPackage{unicode}{hyperref}
\PassOptionsToPackage{hyphens}{url}
%
\documentclass[
]{article}
\usepackage{amsmath,amssymb}
\usepackage{iftex}
\ifPDFTeX
  \usepackage[T1]{fontenc}
  \usepackage[utf8]{inputenc}
  \usepackage{textcomp} % provide euro and other symbols
\else % if luatex or xetex
  \usepackage{unicode-math} % this also loads fontspec
  \defaultfontfeatures{Scale=MatchLowercase}
  \defaultfontfeatures[\rmfamily]{Ligatures=TeX,Scale=1}
\fi
\usepackage{lmodern}
\ifPDFTeX\else
  % xetex/luatex font selection
\fi
% Use upquote if available, for straight quotes in verbatim environments
\IfFileExists{upquote.sty}{\usepackage{upquote}}{}
\IfFileExists{microtype.sty}{% use microtype if available
  \usepackage[]{microtype}
  \UseMicrotypeSet[protrusion]{basicmath} % disable protrusion for tt fonts
}{}
\makeatletter
\@ifundefined{KOMAClassName}{% if non-KOMA class
  \IfFileExists{parskip.sty}{%
    \usepackage{parskip}
  }{% else
    \setlength{\parindent}{0pt}
    \setlength{\parskip}{6pt plus 2pt minus 1pt}}
}{% if KOMA class
  \KOMAoptions{parskip=half}}
\makeatother
\usepackage{xcolor}
\usepackage[margin=1in]{geometry}
\usepackage{graphicx}
\makeatletter
\def\maxwidth{\ifdim\Gin@nat@width>\linewidth\linewidth\else\Gin@nat@width\fi}
\def\maxheight{\ifdim\Gin@nat@height>\textheight\textheight\else\Gin@nat@height\fi}
\makeatother
% Scale images if necessary, so that they will not overflow the page
% margins by default, and it is still possible to overwrite the defaults
% using explicit options in \includegraphics[width, height, ...]{}
\setkeys{Gin}{width=\maxwidth,height=\maxheight,keepaspectratio}
% Set default figure placement to htbp
\makeatletter
\def\fps@figure{htbp}
\makeatother
\setlength{\emergencystretch}{3em} % prevent overfull lines
\providecommand{\tightlist}{%
  \setlength{\itemsep}{0pt}\setlength{\parskip}{0pt}}
\setcounter{secnumdepth}{-\maxdimen} % remove section numbering
\ifLuaTeX
  \usepackage{selnolig}  % disable illegal ligatures
\fi
\usepackage{bookmark}
\IfFileExists{xurl.sty}{\usepackage{xurl}}{} % add URL line breaks if available
\urlstyle{same}
\hypersetup{
  pdftitle={README},
  hidelinks,
  pdfcreator={LaTeX via pandoc}}

\title{README}
\author{}
\date{\vspace{-2.5em}2025-02-21}

\begin{document}
\maketitle

\section{Doku}\label{doku}

\subsection{Vorbereitungen}\label{vorbereitungen}

\begin{itemize}
\item
  Das Geopackage mit Gebäuden (``Gebäude-DB'') wurde von LAEA Europe
  (EPSG 3035) auf Lambert Austria (EPSG 31287) umprojiziert, um der
  Projektion der Eingangsraster (DOM, Globalstrahlung) zu entsprechen.
\item
  Durch Verschneidung der Gebäude-DB mit Gemeindepolygonen wurden eine
  Nachschlagetabelle für die spätere Zuordnung von Gebäuden und
  Gemeinden (und die Aggregierung auf Gemeindeebene) erstellt.
\end{itemize}

\subsection{Konstanten}\label{konstanten}

Die Liste am Eingang des Hauptskripts \texttt{main.R} enthält alle user
servicable parts: Neigungslimit für Flachdächer, Nutzbarkeitsklassen
etc. Die Einstellung der Parameter wird (nur) hier vorgenommen:

\begin{verbatim}
constants <- list(
  flat = 10,  # Schwellenwert (°), unter dem Dach als flach angenommen wird
  steep = 70, # Schwellenwert (°), oberhalb dessen eine Dachfläche als Kante
  ## bzw. nicht montagetauglich betrachtet wird.
  minsize = 3, # erforderliche Mindestausdehnung zusammenhängender Flächen (Pixel = m²)
  # zusammenhängender Dachfläche
  minbuildings = 3, ## Mindestanzahl an Gebäuden, ab der die Kachel berechnet wird
  a_usable = .7,  # Anteil der für PV nutzbaren Dachfläche (0-1)
  modul_m2 = 2.1,  # Fläche pro Modul [m2]
  pv_e = .18,  # PV efficiency (0-1)
  pv_e_f = \(irr_global) .1898 * irr_global - 3.9931, ## Regression statt Konstante
  st_e = .4,  # ST efficiency (0-1)
  buffer = 0,  # Puffer um Gebäudepolygone [m]; nicht puffern, die Berechnung von Neigung/Aspekt 
  ## entfernt sowieso schon den Zellsaum;
  intervals_solar = c(0, 550, 700, 850, 1000, 1150, Inf), ## Klassen solar
  labels = list(
    aspect = c('N', 'NO', 'O', 'SO', 'S', 'SW', 'W', 'NW'),
    eignung_solar = c('nicht', 'wenig_2040', 'wenig_2020', 'geeignet', 'gut', 'sehr_gut')
    )
)
\end{verbatim}

\subsection{Methodik}\label{methodik}

\subsubsection{Workflow}\label{workflow}

\paragraph{Teil 1: von den Eingangsdateien bis zur
Hauptmaske}\label{teil-1-von-den-eingangsdateien-bis-zur-hauptmaske}

Die in der Gebäude-DB enthaltenen Gebäudegeometrien werden nur für die
jeweils bearbeitete Kachel abgefragt und in ein Raster mit Auflösung und
Projektion der DOM-Kachel umgewandelt. Dieses Raster enthält OBJECTID
und Abgrenzung der Gebäude für die nachfolgenden Berechnungen.

Aus dem DOM wird die Neigung (slope) abgeleitet, aus der u. a. die
Hauptmaske generiert wird. Die Hauptmaske scheidet alle Rasterpixel aus,
die bei der Auswertung vernachlässigt werden sollen. Dazu gehören:

\begin{itemize}
\tightlist
\item
  Pixel außerhalb von Gebäuden (Objekten) lt. Gebäude-DB
\item
  Grate: langgezogene, dünne (≤ \texttt{minsize} Pixelreihen) mit max.
  10° Neigung, wie sie im DOM z. B. durch Dachfirste erzeugt werden
\item
  Inseln: kleine Pixelhaufen, etwa durch Dachaufbauten, mit den
  Ausscheidungskriterien für Grate
\item
  Kanten oder steile Flächen \textgreater{} \texttt{steep}
  Neigungswinkel
\end{itemize}

Mit dieser Hauptmaske werden die übrigen Eingangsdaten maskiert, d.~h.
ausgeschiedene Pixel werden auf rechnerisch neutral (\texttt{NA})
gesetzt.

\begin{itemize}
\item
  Neigung und Aspekt werden aus den 4 orthogonal angrenzenden Pixeln
  berechnet (nicht wie ursprünglich aus den 8 umrandenden). Dadurch
  reichen auch die Neigungs- und Aspektzellen bis an den Dachrand (davor
  fiel ein Saum von 1 Pixelbreite am Dachrand als NA aus)
\item
  Vorschlag: Neigung zum Beschneiden der Dachfläche verwenden
\end{itemize}

\begin{verbatim}
## file:////tmp/RtmpgXEuf4/file2c0845d467e3a/widget2c0843b93ebc7.html screenshot completed
\end{verbatim}

\includegraphics{README_files/figure-latex/unnamed-chunk-1-1.pdf}

\subsubsection{Ausscheiden zu kleiner
Dachflächen}\label{ausscheiden-zu-kleiner-dachfluxe4chen}

Die Ausscheidung zu kleiner Dachflächen erfolgt dzt. nicht mit einem
rechteckigen moving window, sondern über die Größe zusammenhängender
Dachpixel (m²). Damit werden auch unregelmäßige Formen (``Tetris-Blöcke
erfasst''). Grund ist der drastische Geschwindigkeitsunterschied
(Millisekunden vs.~Sekunden) zwischen den Berechnungsarten. Da die
Dächer ohnehin häufig schräg zum Gitter orientiert sind, würde auch das
rechteckige moving window die Wirklichkeit nicht unbedingt besser
wiedergeben.

\subsection{Ergebnisse}\label{ergebnisse}

\begin{verbatim}
## file:////tmp/RtmpgXEuf4/file2c0844f824fac/widget2c08449565f23.html screenshot completed
\end{verbatim}

\includegraphics{README_files/figure-latex/unnamed-chunk-2-1.pdf}

\subsubsection{Einfluss Pufferdistanz}\label{einfluss-pufferdistanz}

Bei ca. 1,5 m Puffer (Erweiterung des Gebäudeumrissen treten die meisten
Höhen-Ausreißer auf, von da weg nehmen sie rel. linear in beide
Richtungen ab. Eine markante Änderung der Ausreißerzahl bei einer
bestimmten Pufferdistanz zur Optimierung der Distanz ist leider nicht
erkennbar.)

\subsubsection{Höhenausreißer}\label{huxf6henausreiuxdfer}

Die jetzige Ausreißerdefinition würde Flachdächer desselben Gebäudes bei
größeren Niveauunterschieden ausscheiden:

\subsubsection{Kontrolle Aspekt}\label{kontrolle-aspekt}

Unterscheidung nach acht Himmelsrichtungen am Beispiel der Domkuppel
(Salzburg, Kachel 27475-45475)

\end{document}
